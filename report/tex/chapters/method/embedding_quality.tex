\section{Embedding quality}\label{chap:embedding_quality}
\paragraph{}The quality of embeddings is paramount in machine learning, particularly when the objective is to identify specific \glspl{chunk} within data, such as the ones holding SSH keys. It becomes essential to juxtapose the performances of all embeddings in this context. An optimal embedding should proficiently discern the \glspl{chunk} containing SSH keys across the entire spectrum of openSSH use cases and for every conceivable key size. This necessitates the utilization of the complete dataset, with the training subset dedicated to model training and the validation subset for testing. Addressing this from a machine learning classification perspective, the random forest model, as elucidated in \ref{seq:background:some_ml_common_models}, emerges as the classifier of choice.

\paragraph{}To ensure fairness and comparability among the embeddings, we employ the Pearson correlation method \ref{seq:background:correlation_tests} to limit the selection to the top 8 correlations, thereby narrowing down our analysis to the most influential features. The dataset is notably imbalanced \ref{seq:background:imbalanced_data}, primarily stemming from the rarity of memory structures containing SSH keys, our specific target of interest, within the overall dataset. This rarity results in a significant class imbalance, where the majority of memory structures do not contain SSH keys. To counteract potential bias toward the majority class, we will implement random undersampling as a resampling strategy, particularly given our very large dataset. However, we will refrain from using undersampling when the chunk size or the entropy filter are active to maintain the integrity of the filtered data. This approach will enable our model to accurately classify both majority and minority classes without being overwhelmed by the sheer volume of data. We will then employ a Random Forest model \ref{seq:background:machine_learning}, renowned for its robustness and suitability for high-dimensional data, to carry out the classification task. Our evaluation will rely on metrics such as precision, recall, F1 score, and others to identify the most effective representation for precise classification.

\subsection{Feature Selection and Dataset Challenges}

\paragraph{}In the quest for fairness across various embeddings and to circumvent the curse of dimensionality, it's imperative to maintain a uniform feature count across all embeddings. This is where feature engineering shines. The Pearson correlation method, elaborated in \ref{seq:background:correlation_tests}, is harnessed to meticulously select the 8 most salient features for each embedding. This count is a judicious compromise, ensuring the features are both succinct in number and information-rich. However, the dataset presents its own set of challenges. The instances of \glspl{chunk} containing SSH keys are dwarfed by those devoid of them, leading to a pronounced dataset imbalance. To counteract this skewness, the random undersampling technique, as referenced in \ref{seq:background:imbalanced_data}, is employed, when the dataset isn't filtered.

\subsection{Implementation and Evaluation Metrics}

\paragraph{}The implementation leans heavily on the scikit-learn library \cite{pedregosa_scikit-learn_2011} in Python, which provides the tools for the random forest classifier, Pearson correlation, and the random undersampling algorithm. Concurrently, the pandas library is indispensable for the efficient loading and manipulation of the dataset. Before diving into the analysis, it's crucial to ensure the embedding's integrity. This involves a rigorous sanity check, especially given the potential for corruption, such as NaN values. To guarantee the reproducibility of results, a consistent random seed is employed for both the random forest classifier and the random undersampling algorithm. For a comprehensive evaluation, the Pearson correlation matrix is preserved for each embedding. Moreover, a suite of metrics, including precision, recall, f1-score, AUC, and the confusion matrix (encompassing true positives, true negatives, false positives, and false negatives), is meticulously saved for every embedding.
