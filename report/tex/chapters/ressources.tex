\chapter{Ressources}\label{chap:ressources}

\paragraph{}In the resources chapter of this thesis, we meticulously detail the various materials and tools that have been utilized throughout the course of the research. 

\section{hardware}
\paragraph{}My primary workstation is an \textit{Aspire 5} laptop, equipped with:
\begin{itemize}
    \item \textbf{CPU:} 11th Gen Intel i5-1135G7 (8) @ 4.200GHz 
    \item \textbf{GPU:} Intel TigerLake-LP GT2 [Iris Xe Graphics]
    \item \textbf{Memory:} 16GB
\end{itemize}
\paragraph{}However, this laptop, despite its decent specifications, proved inadequate for processing the entire dataset. Simple machine learning experiments using a Python script would have stretched over a week. Even when we transitioned to more optimized Rust programs, the processing time exceeded 10 hours. While we managed to run minor tasks and scripts on this laptop, the bulk of the experiments necessitated a more powerful server.

\paragraph{}Recognizing this need, we was granted access to a high-performance development server in the later stages of the thesis, around August 2023. The server, an \textit{AS-4124GS-TNR}, boasts the following specifications:
\begin{itemize}
    \item \textbf{CPU:} 2x AMD EPYC 7662 (256) @ 2.000GHz
    \item \textbf{GPU:} NVIDIA Geforce RTX 3090 Ti
    \item \textbf{RAM:} 512GB DDR4 3200MHz
\end{itemize}
\paragraph{}Operating on \textit{Ubuntu 20.04.6 LTS}, this server became the primary platform for the machine learning experiments, given its superior computational capabilities compared to the \textit{Aspire 5} laptop. This invaluable resource was generously provided by the Department of Computer Science at \textit{Universität Passau}, particularly under the guidance of the Chair of Data Science led by Prof. Dr. Michael Granitzer. I extend my sincere appreciation for their unwavering support.

\section{Code Repository}

\paragraph{}The code repository section of this thesis is dedicated to providing open access to all the data and code used throughout the research. Hosted on GitHub under the organization \textit{Passau Masterarbeit 2023}, the repositories are a testament to the commitment to transparency and collaborative development.

\paragraph{}The repository \textit{mem2graph}\footnote{\url{https://github.com/passau-masterarbeit-2023/mem2graph}} is particularly noteworthy as it contains the code for analyzing memory dumps and converting them into simple embeddings. This repository is a crucial resource for anyone looking to understand or replicate the memory dump analysis process.

\paragraph{}For those interested in the deep learning and machine learning aspects of the thesis, the repository \textit{phdtrack\_openssh\_memory\_embedding}\footnote{\url{https://github.com/passau-masterarbeit-2023/phdtrack_openssh_memory_embedding}} is invaluable. It not only includes the code for these processes but also houses the complete results of the experiments in the \texttt{results\_server\_full/} folder, providing a thorough record of the research findings.

\paragraph{}Lastly, the repository \textit{phdtrack\_server\_scripts}\footnote{\url{https://github.com/passau-masterarbeit-2023/phdtrack_server_scripts}} offers the scripts used to set up the environment on the server, ensuring that the computational setup can be replicated or adapted for future research endeavors.

