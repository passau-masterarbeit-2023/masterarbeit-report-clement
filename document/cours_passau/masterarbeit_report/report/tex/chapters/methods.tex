\chapter{Methods}\label{chap:methods}
    \paragraph{}This research dives into the complexities of embedding byte sequences, focusing particularly on the extraction of structures containing SSH keys for machine learning purposes. The varied uses of OpenSSH introduce distinct challenges due to potential variations in the created embeddings. Given the wide array of SSH key dimensions and OpenSSH's intricate operations, maintaining the embeddings' stability and consistency is vital. In this methodological section, we will detail various embedding methods, present a framework for their assessment through a classifier model, and suggest another strategy to verify the embeddings' coherence between the different OpenSSH usage and key sizes.

    \section{hardware}
    

    \section{Dataset}
        \paragraph{}The dataset at the core of this thesis, as previously introduced (see \ref{seq:background:dataset}), consists of heap dump raw files related to different OpenSSH use cases and versions. Each heap dump file is paired with a JSON annotation file created by the dataset's creators. These JSON files provide extra information about the heap dump, especially regarding encryption keys. In this section, we will explain our exploration of the dataset, aiming to better comprehend its content and nuances.

        \subsection{Estimating the dataset balancing for key prediction}
            \paragraph{}In this part, our primary objective was to assess the balance of the dataset for key prediction and identify the challenges associated with it.

            \paragraph{}To begin, we aimed to gain an understanding of the dataset's scale. We utilized a code snippet \ref{methods:code:count_all_dataset_files} to count all the files within the dataset, revealing a total of 208,745 files. However, it was imperative to recognize that JSON files, which served as annotation files, were not to be considered part of the raw bytes for embedding. Consequently, these JSON files were excluded from our count to provide a more accurate representation of the dataset's size.

            \begin{lstlisting}[caption={Count all dataset files}, label=methods:code:count_all_dataset_files, language=bash]
            find . -type f | wc -l
            \end{lstlisting}

            \paragraph{}Following this, we employed another code snippet \ref{methods:code:count_raw_files} to specifically count the heap dump raw files, excluding JSON files. This count indicated a total of 103,595 heap dump raw files, which constituted the primary focus of our analysis.

            \begin{lstlisting}[caption={Count heap dump raw dataset files}, label=methods:code:count_raw_files, language=bash]
            find . -type f -name "*.raw" | wc -l
            \end{lstlisting}

            \paragraph*{}To gain further insights into the dataset, we determined its size while excluding annotation files \ref{methods:code:get_dataset_size}. The calculated dataset size amounted to 18,067,001,344 bytes.

            \begin{lstlisting}[caption={Get the size of the dataset}, label=methods:code:get_dataset_size, language=bash]
            find . -type f -name "*.raw" -exec du -b {} + | awk '{s+=$1} END {print s}'
            \end{lstlisting}

            \paragraph{}Considering the nature of the dataset, which featured a maximum of six keys per file, each with a maximum size of 64 bytes, we conducted a rough estimate. We determined that the maximum number of bytes relevant for searching across the dataset was $6 * 64 * 103595 = 39 780 480$ . This calculation accounted for approximately 0.22\% of the dataset's total size.

            \paragraph{}Lastly, it is crucial to acknowledge that the dataset exhibited a significant imbalance and is very large. To address this challenge effectively, strategies were implemented to ensure robust, unbiased analyses, and scalability.
        \subsection*{Annotations}
            \paragraph{}The annotations files are essential to understand the data and how best to utilize them for the study. Each heap dump corresponds to one specific JSON file. To view the contents of these JSON files in a more organized manner, one can reference the method provided at \ref{methods:code:pretty_print_json}. For a clearer understanding, an extract of the JSON annotation from the file located at \path{./Training/client/V_7_8_P1/16/13116-1644920217.json} is available at \ref{methods:code:annotation_extract}.

            \begin{lstlisting}[caption={pretty print JSON}, label=methods:code:pretty_print_json, language=bash]
                python3 -m json.tool file.json
            \end{lstlisting}
            \begin{lstlisting}[language=json, caption={An extract of the JSON annotations}, label=methods:code:annotation_extract]
            {
                /* file ./Training/client/V_7_8_P1/16/13116-1644920217.json*/
                "SSH_STRUCT_ADDR": "5619dd7e5570",
                "SESSION_STATE_ADDR": "5619dd7e5df0",
                "KEY_A_ADDR": "5619dd807f40",
                "KEY_A_LEN": "12",
                "KEY_A_REAL_LEN": "12",
                "KEY_A": "34fbe182e76c49a617a93e2e",
                /*...*/
                "KEY_E_ADDR": "5619dd808000",
                "KEY_E_LEN": "0",
                "KEY_E_REAL_LEN": "0",
                "KEY_E": "",
                "KEY_F_ADDR": "5619dd807fd0",
                "KEY_F_LEN": "0",
                "KEY_F_REAL_LEN": "0",
                "KEY_F": "",
                "HEAP_START": "5619dd7e3000"
            }
            \end{lstlisting}

            \paragraph{}Within these annotation files, several critical pieces of information are present. The ``SSH\_STRUCT\_ADDR'' and ``SESSION\_STATE\_ADDR'' denote the addresses of vital openSSH \glspl{structure}. These addresses are pivotal in gauging the embedding coherence across different openSSH uses and key sizes. If the embeddings of these \glspl{structure} display similarity across various key sizes and openSSH usages, it signifies the embedding's coherence.

            \paragraph{}Other significant annotations such as ``KEY\_A\_ADDR'', ``KEY\_A\_LEN'', ``KEY\_A\_REAL\_LEN'', and ``KEY\_A'' detail the address, length, and value of the key A. In general, six of these annotations can be found for each heap dump. Notably, the ``HEAP\_START'' annotation, along with the length of the heap dump, is of paramount importance. This annotation signifies the starting address of the heap dump. This information not only aids in pinpointing addresses in the heap dump for \glspl{structure} and \glspl{pointer}, but also refines the heuristic used in detecting \glspl{pointer} \ref{seq:background:pointers_and_structures}. By leveraging the ``HEAP\_START'' information, one can verify if a \gls{pointer} is pointing within the heap dump boundaries. As a practical illustration, deducing the address of key A within the heap dump can be achieved by subtracting ``HEAP\_START'' from ``KEY\_A\_ADDR''.

            \paragraph{}However, it's noteworthy that some of these annotation files may be corrupted. Therefore, it's imperative to verify the integrity of each file before its use. In instances where keys are corrupted, such as "KEY\_E" and "KEY\_F" having no recorded values in the extract found at \ref{methods:code:annotation_extract}, it's advised either to remove the corrupted keys or discard the entire file if the data cannot be salvaged. Armed with this understanding, the next logical step would be to leverage this dataset to formulate embeddings and subsequently evaluate their performance.

        \subsection{Particularities of heap dumps}
            **wait dataset information**

        
    \section{Embeddings}
        \paragraph{}From the Zenodo dataset\ref{seq:background:dataset}, we've isolated distinct memory structures within the raw heap dump files. These structures possess diverse sizes, necessitating the use of an embedding method for classification. Fortunately, a distinguishing feature of each memory structure is the presence of a header, containing vital information such as the structure's size in bytes. To precisely pinpoint the boundaries of each memory structure, we sequentially parse through the raw heap dump files. Beginning the parsing process from the first non-null byte, identified as the header, serves as a marker for the initiation of a new structure. The size data within this header is then leveraged to calculate the exact length of the structure, allowing for the extraction of its entire raw byte data while determining the start of the subsequent one.
        
        \paragraph{}Our next objective centers on the conversion of raw byte data into fixed-size embeddings (\ref{seq:background:traditional_statistical_embedding}, \ref{seq:background:deep_learning_models_for_raw_byte_embedding}), a pivotal step in preparing them for utilization in machine learning applications. Ensuring uniformity in embedding size across all memory structures holds paramount significance. Consistency in embedding dimensions is vital to empower machine learning algorithms for efficient data processing and analysis. This uniformity not only simplifies the integration of memory structures with varying sizes into a coherent classification framework but also acts as a defense against the adverse effects of the curse of dimensionality—a phenomenon that can introduce computational complexities and heighten the risk of overfitting in high-dimensional data spaces. Striking this equilibrium is essential, achieved by maintaining reasonably low embedding dimensions, fostering both efficient data processing and the preservation of essential information within the raw byte data. It's important to note that initially, each embedding will include the structure's file and the structure's address in the file. However, these details will be removed during the machine learning phase (quality or coherence) as the embedding aims to be free of key size or OpenSSH uses. Their presence will serve as a means to test coherence later in our analysis.

    \section{Embedding quality}
        \paragraph{} Transitioning our focus, we now delve into evaluating the quality of the embeddings. To ensure fairness and comparability among the embeddings, we employ the Pearson correlation method \ref{seq:background:correlation_tests} to limit the selection to the top 8 correlations, thereby narrowing down our analysis to the most influential features. The dataset is notably imbalanced \ref{seq:background:imbalanced_data}, primarily stemming from the rarity of memory structures containing SSH keys, our specific target of interest, within the overall dataset. This rarity results in a significant class imbalance, where the majority of memory structures do not contain SSH keys. To counteract potential bias toward the majority class, we will implement the \acrfull{smote} as a resampling strategy, enabling our model to accurately classify both majority and minority classes. We will then employ a Random Forest model \ref{seq:background:machine_learning}, renowned for its robustness and suitability for high-dimensional data, to carry out the classification task. Our evaluation will rely on metrics such as precision, recall, F1 score, and others to identify the most effective representation for precise classification.

    \section{Embedding coherence}
        \paragraph{}After completing the classification task, our focus shifts to evaluating the coherence of the embedding across different applications of OpenSSH and various key sizes. To accomplish this, we will utilize a clustering model, specifically DBSCAN, which is well-suited for scenarios where the number of clusters is uncertain. Our objective is to determine if the formed clusters demonstrate coherence, signifying the proximity of memory structures containing SSH keys. This analysis also encompasses an assessment of the underlying embedding method's consistency across various uses of SSH and key sizes, illustrating its ability to capture significant patterns and relationships related the the SSH keys.

        \paragraph{}In the following section, we will delve deep into the methodologies and techniques utilized to construct these embeddings, offering a comprehensive insight into the fundamental building blocks of our study.