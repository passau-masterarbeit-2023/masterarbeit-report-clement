\chapter{Methods}\label{chap:methods}
    \paragraph{}This research dives into the complexities of embedding byte sequences, focusing particularly on the extraction of structures containing SSH keys for machine learning purposes. The varied uses of OpenSSH introduce distinct challenges due to potential variations in the created embeddings. Given the wide array of SSH key dimensions and OpenSSH's intricate operations, maintaining the embeddings' stability and consistency is vital. In this methodological section, we will detail various embedding methods, present a framework for their assessment through a classifier model, and suggest another strategy to verify the embeddings' coherence between the different OpenSSH usage and key sizes.

    
    \section{Embeddings}
    \paragraph{}From the Zenodo dataset\ref{seq:background:dataset}, we've isolated distinct memory structures within the raw heap dump files. These structures possess diverse sizes, necessitating the use of an embedding method for classification. Fortunately, a distinguishing feature of each memory structure is the presence of a header, containing vital information such as the structure's size in bytes. To precisely pinpoint the boundaries of each memory structure, we sequentially parse through the raw heap dump files. Beginning the parsing process from the first non-null byte, identified as the header, serves as a marker for the initiation of a new structure. The size data within this header is then leveraged to calculate the exact length of the structure, allowing for the extraction of its entire raw byte data while determining the start of the subsequent one.
    
    \paragraph{}Our next objective centers on the conversion of raw byte data into fixed-size embeddings (\ref{seq:background:traditional_statistical_embedding}, \ref{seq:background:deep_learning_models_for_raw_byte_embedding}), a pivotal step in preparing them for utilization in machine learning applications. Ensuring uniformity in embedding size across all memory structures holds paramount significance. Consistency in embedding dimensions is vital to empower machine learning algorithms for efficient data processing and analysis. This uniformity not only simplifies the integration of memory structures with varying sizes into a coherent classification framework but also acts as a defense against the adverse effects of the curse of dimensionality—a phenomenon that can introduce computational complexities and heighten the risk of overfitting in high-dimensional data spaces. Striking this equilibrium is essential, achieved by maintaining reasonably low embedding dimensions, fostering both efficient data processing and the preservation of essential information within the raw byte data. It's important to note that initially, each embedding will include the structure's file and the structure's address in the file. However, these details will be removed during the machine learning phase (quality or coherence) as the embedding aims to be free of key size or OpenSSH uses. Their presence will serve as a means to test coherence later in our analysis.

    \section{Embedding quality}
    \paragraph{} Transitioning our focus, we now delve into evaluating the quality of the embeddings. To ensure fairness and comparability among the embeddings, we employ the Pearson correlation method \ref{seq:background:correlation_tests} to limit the selection to the top 8 correlations, thereby narrowing down our analysis to the most influential features. The dataset is notably imbalanced \ref{seq:background:imbalanced_data}, primarily stemming from the rarity of memory structures containing SSH keys, our specific target of interest, within the overall dataset. This rarity results in a significant class imbalance, where the majority of memory structures do not contain SSH keys. To counteract potential bias toward the majority class, we will implement the \acrfull{smote} as a resampling strategy, enabling our model to accurately classify both majority and minority classes. We will then employ a Random Forest model \ref{seq:background:machine_learning}, renowned for its robustness and suitability for high-dimensional data, to carry out the classification task. Our evaluation will rely on metrics such as precision, recall, F1 score, and others to identify the most effective representation for precise classification.

    \section{Embedding coherence}
    \paragraph{}After completing the classification task, our focus shifts to evaluating the coherence of the embedding across different applications of OpenSSH and various key sizes. To accomplish this, we will utilize a clustering model, specifically DBSCAN, which is well-suited for scenarios where the number of clusters is uncertain. Our objective is to determine if the formed clusters demonstrate coherence, signifying the proximity of memory structures containing SSH keys. This analysis also encompasses an assessment of the underlying embedding method's consistency across various uses of SSH and key sizes, illustrating its ability to capture significant patterns and relationships related the the SSH keys.

    \paragraph{}In the following section, we will delve deep into the methodologies and techniques utilized to construct these embeddings, offering a comprehensive insight into the fundamental building blocks of our study.