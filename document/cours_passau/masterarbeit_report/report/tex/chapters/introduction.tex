\chapter{Introduction}\label{chap:introduction}


\begin{comment}
Motivate your research and outline the research gap in this chapter. Why is your thesis relevant and what do you address, what has not been addressed before. 

General Requirements to the thesis:

\begin{itemize}
	\item 60 pages of content in this format. Content does not include table of content, lists, appendices etc.
	\item Proper scientific referencing
	\item Introduction and Background should be less than 50\% of the thesis
	\item Images should be readable and in the proper size. 
\end{itemize}
\end{comment}


\paragraph*{}Digital forensics is a linchpin in cybersecurity, enabling the extraction of vital evidence from devices like PCs. This evidence is key for detecting malware and tracing intruder activities. Analyzing a device's main memory is a go-to technique in this field. The fusion of machine learning promises to amplify and streamline these analyses.
 
\paragraph*{}With the rising need for encrypted communication, \acrfull{ssh} protocols are now commonplace. However, these security-focused channels can inadvertently shield malicious actions, posing challenges to standard investigative approaches. Cutting-edge research offers solutions. The work in \citetitle*{fellicious_smartkex_2022}~\cite*{fellicious_smartkex_2022} highlights how machine learning can boost the extraction of session keys from OpenSSH memory images. In a complementary vein, \citetitle*{sentanoe_sshkex_2022}~\cite*{sentanoe_sshkex_2022} showcases the power of \acrfull{vmi} for direct SSH key extraction.

\paragraph*{}Inspired by \citetitle*{fellicious_smartkex_2022}, this thesis zeroes in on a central challenge: data embedding. While previous studies set the stage for key extraction, the data embedding technique, especially windowing, can be optimized. The design of data embeddings is pivotal for machine learning efficacy, especially in nuanced tasks like memory analysis. This research introduces fresh embedding strategies, aiming to refine extraction and unearth deeper memory snapshot patterns. Merging graph embeddings with advanced machine learning, the goal is to craft a sophisticated toolkit for OpenSSH heap dump studies, bridging digital forensics and machine learning.


\chapter{Research Questions}

\paragraph*{}A primary focus is on identifying the most suitable techniques for embedding byte sequences, particularly when the objective is to extract structures housing SSH keys for machine learning applications. As the research progresses, it becomes essential to discern if the designed embeddings manifest distinct variations depending on the diverse applications of OpenSSH. Given the extensive spectrum of SSH key sizes and the intricate workings of OpenSSH, ensuring the consistency and stability of these embeddings is paramount. Addressing these aspects will pave the way for a deeper comprehension of challenges and prospective solutions in the field of data embedding, aligning with the overarching goal of bridging digital forensics and machine learning.

\chapter{Structure of the Thesis}

Explain the structure of the thesis. 
