\section{Introduction}\label{chap:introduction}

\begin{comment}
Motivate your research and outline the research gap in this chapter. Why is your thesis relevant and what do you address, what has not been addressed before. 

General Requirements to the thesis:

\begin{itemize}
	\item 60 pages of content in this format. Content does not include table of content, lists, appendices etc.
	\item Proper scientific referencing
	\item Introduction and Background should be less than 50\% of the thesis
	\item Images should be readable and in the proper size. 
\end{itemize}
\end{comment}

\paragraph*{}Digital forensics stands as a cornerstone of cybersecurity and investigation. It provides the means to retrieve crucial evidence from devices, including personal computers. Such evidence is instrumental in identifying malware or tracing the digital footprints of potential intruders. A predominant technique in this domain involves analyzing the contents of a device's primary memory. The integration of machine learning offers a promising avenue to enhance and refine these analysis processes.
\paragraph*{}Moreover, As the demand for secure communication channels grows, protocols like Secure Shell (\acrshort{ssh}) have become ubiquitous. However, these very channels, designed for security, can sometimes obscure malicious activities, making traditional investigative methods less effective. Recent research has highlighted innovative approaches to address these challenges. For instance, the study on \citetitle*{fellicious_smartkex_2022}~\cite*{fellicious_smartkex_2022} underscores the potential of machine learning in enhancing the extraction of session keys from memory snapshots of OpenSSH processes. Furthering this line of inquiry, another pivotal work titled \citetitle*{sentanoe_sshkex_2022}~\cite*{sentanoe_sshkex_2022} introduced the concept of leveraging Virtual Machine Introspection (\acrshort{vmi}) for extracting SSH's session keys directly from a server's memory. 

\section{Research Questions}

Write down and explain your research questions (2-5)

\section{Structure of the Thesis}

Explain the structure of the thesis. 

\section{Example citation \& symbol reference}\label{sec:citation}
For symbols look at.


\section{Example reference}
Example reference: Look at chapter~\ref{chap:introduction}, for sections, look at section~\ref{sec:citation}.

\section{Example image}

\begin{figure}
	\centering
	\includegraphics[width=0.5\linewidth]{uni-logo}
	\caption{Meaningful caption for this image}
	\label{fig:uniLogo}
\end{figure}

Example figure reference: Look at Figure~\ref{fig:uniLogo} to see an image. It can be \texttt{jpg}, \texttt{png}, or best: \texttt{pdf} (if vector graphic).

\section{Example table}

\begin{table}
	\centering
	\begin{tabular}{lr}
		First column & Number column \\
		\hline
		Accuracy & 0.532 \\
		F1 score & 0.87
	\end{tabular}
	\caption{Meaningful caption for this table}
	\label{tab:result}
\end{table}

Table~\ref{tab:result} shows a simple table\footnote{Check \url{https://en.wikibooks.org/wiki/LaTeX/Tables} on syntax}